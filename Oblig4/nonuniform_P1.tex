\documentclass[12pt, a4paper]{article}

\usepackage{amsmath}
\usepackage{graphicx}
\usepackage{float}
\usepackage{listings}
\usepackage{color}

\lstset{
	numbers=left, 
	numberstyle=\small, 
	numbersep=8pt, 
	frame = single, 
	language=Python, 
	framexleftmargin=15pt,
	breaklines,
	showspaces=false, 
	showstringspaces=false, 
	backgroundcolor=\color{white}, 
	keywordstyle=\color{red}\bfseries, 
	showtabs=false} 

\usepackage[norsk]{babel}
\usepackage{csquotes}
\setlength {\marginparwidth }{2cm}
\usepackage{todonotes} 
\usepackage{subcaption}
\usepackage{comment}
\usepackage{wrapfig}
\usepackage[margin=2cm]{geometry}
\usepackage{amsthm}

\title{Exercise 8 - Compute with a nonuniform mesh}
\date{October 2019}
\author{Marte Fossum}

\begin{document}

\maketitle
\newpage

{\setlength{\parindent}{0cm}

\section*{8a}

We are given the problem

\begin{equation} \label{main}
    -u''(x) = 2
\end{equation}

for \(x \in [0,1]\) with \(u(0) = 0\) and \(u(1) = 1\), and we want to derive a linear system using P1 elements with a non-uniform mesh \(x_{0}=0<x_{1}< {...} <x_{N_{n}-1}=1\). \\

We seek a solution \(u \in V\), \(V = span\{\psi_0(x), ..., \psi_{N_n - 1}(x)\}\), where \(\{\psi_i\}\) is a set of linearly independent basis functions for \(i \in \{0, ..., N_n -1\}\). Since \(u \in V\) we can write \(u\) as 

\begin{equation} \label{u_sum}
    u(x) = \sum_{i = 0}^{N_{n}-1} c_i\psi_i(x)
\end{equation}

We take the innerproduct of (\ref{main}) on both sides with some function \(v \in V\). Letting \(v = \psi_j\) gives us 

\begin{align*}
    \langle 2, \psi_j(x)\rangle &= \langle -u''(x), \psi_j(x)\rangle \\
    \int_{0}^{1} 2\psi_j(x) dx &= \int_{0}^{1} -u''(x)\psi_j(x) dx \\
    h_{j-1} + h_j &= -[u'(x)\psi_j(x)]_0^1 + \int_0^1 u'(x)\psi_j'(x) dx \\
    &= \int_0^1 u'(x)\psi_j'(x) dx \\
    &= \int_0^1 \sum_{i = 0}^{N_n -1} c_i \psi_i'(x)\psi_j'(x) dx \\
    &= \sum_{i = 0}^{N_n - 1} c_i \int_0^1 \psi_i'(x) \psi_j'(x) dx
\end{align*}

Since \(\psi_j\) is zero everywhere but on the interval \([x_{j-1}, x_{j+1}]\), we have that the derivative is also zero everywhere except on the same interval. More precisely 

\begin{equation*}
    \psi_j'(x) = 
    \begin{cases}
        \frac{1}{h_{j-1}} & x \in (x_{j-1}, x_j) \\
        -\frac{1}{h_{j}} & x \in (x_{j}, x_{j+1})\\
        0 & \text{otherwise}
    \end{cases}
\end{equation*}

Put together we get that

\begin{align*}
    h_{j-1} + h_j &= c_{j-1}A_{j-1,j} + c_jA_{j,j} + c_{j+1}A_{j+1,j}
\end{align*}

where 

\begin{align*}
    A_{j-1,j} &= \int_{x_{j-1}}^{x_j} \frac{-1}{h_{j-1}^2} dx \\
    &= -\frac{1}{h_j} \\\\
    A_{j,j} &= \int_{x_{j-1}}^{x_j} \frac{1}{h_{j-1}^2} dx + \int_{x_{j}}^{x_{j+1}} \frac{1}{h_{j}^2} dx \\
    &= \frac{1}{h_{j-1}} + \frac{1}{h_j} \\\\
    A_{j+1,j} &= \int_{x_{j}}^{x_{j+1}} \frac{-1}{h_{j}^2} dx \\
    &= -\frac{1}{h_j} \\\\
\end{align*}

This all works for \(j = 1,..., N_n - 2\). However we now only have \(N_n - 2\) equations, but we have \(N\) unknowns. This is because we don't have any equations for the boundary. We do know however that \(u(0) = 0\) and that \(u(1) = 1\) which gives us the last two equations \(c_0 = 0\) and \(c_{N_n -1} = 1\). 

\begin{equation} \label{endelement}
    \begin{cases}
        h_{j-1} + h_j = -\frac{1}{h_j}c_{j+1} + \left(\frac{1}{h_j} + \frac{1}{h_{j-1}}\right)c_j - \frac{1}{h_{j-1}}c_{j-1}& \text{for \(j = 1,...,N_n + 2\)} \\
        c_0 = 0 & \text{for \(j = 0\)} \\
        c_{N_n - 1} = 1 & \text{for \(j = N_n - 1\)} \\
    \end{cases}
\end{equation}

\newpage 

\section*{8b}

We want to use the finite difference method to discretize \(u''(x_i) = [D_x D_x u]_i\) and comepare this to the finite element. 

\begin{align} \label{diff}
    u''(x_i) &= [D_x(D_x u)]_i \nonumber \\ \nonumber \\
            &= \frac{[D_x u]_{i + \frac{1}{2}} - [D_x u]_{i - \frac{1}{2}}}{x_{i + \frac{1}{2}} - x_{i - \frac{1}{2}}} \nonumber \\\nonumber\\
            &= \frac{1}{x_{i + \frac{1}{2}} - x_{i - \frac{1}{2}}}\left(\frac{u_{i+\frac{1}{2} + \frac{1}{2}} - u_{i + \frac{1}{2} - \frac{1}{2}}}{x_{i + \frac{1}{2} + \frac{1}{2}} - x_{i + \frac{1}{2} - \frac{1}{2}}} - \frac{u_{i-\frac{1}{2} + \frac{1}{2}} - u_{i - \frac{1}{2} - \frac{1}{2}}}{x_{i - \frac{1}{2} + \frac{1}{2}} - x_{i - \frac{1}{2} - \frac{1}{2}}}\right) \nonumber \\\nonumber\\
            &= \frac{1}{x_{i + \frac{1}{2}} - x_{i - \frac{1}{2}}}\left(\frac{u_{i+1} - u_i}{x_{i+1} - x_i} - \frac{u_i - u_{i-1}}{x_i - x_{i-1}}\right) \nonumber\\\nonumber\\
            &= \frac{2}{x_{i+1} - x_{i-1}}\left(\frac{u_{i+1} - u_i}{x_{i+1} - x_i} - \frac{u_i - u_{i-1}}{x_i - x_{i-1}}\right) 
\end{align}

We have that each interval between points \(x_i\) and \(x_{i+1}\) has length \(h_i\), so substituting \(h_i = x_{i+1} - x_i\) into (\ref{main}) and (\ref{diff}) gives 

\begin{equation*}
    -u''(x_i) = - \frac{2}{h_{i} + h_{i-1}}\left(\frac{u_{i+1} - u_i}{h_i} - \frac{u_i - u_{i-1}}{h_{i-1}}\right) = 2
\end{equation*}

\begin{align} \label{enddisc}
    \implies h_i + h_{i-1} &= -\frac{u_{i+1} - u_i}{h_i} + \frac{u_i - u_{i-1}}{h_{i-1}} \nonumber \\ \nonumber \\
    &= -\frac{1}{h_i}u_{i+1} + \left(\frac{1}{h_i} + \frac{1}{h_{i-1}}\right)u_i - \frac{1}{h_{i-1}}u_{i-1}
\end{align}

Comparing the end result for the discrete method (\ref{enddisc}) with the result for the finite element method (\ref{endelement}) we can see that they are the same type of linear systems. 

\end{document}