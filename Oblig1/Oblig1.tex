\documentclass[12pt, a4paper]{article}

\usepackage{amsmath}
\usepackage{graphicx}
\usepackage{float}
\usepackage{listings}
\usepackage{color}

\lstset{
	numbers=left, 
	numberstyle=\small, 
	numbersep=8pt, 
	frame = single, 
	language=Python, 
	framexleftmargin=15pt,
	breaklines,
	showspaces=false, 
	showstringspaces=false, 
	backgroundcolor=\color{white}, 
	keywordstyle=\color{red}\bfseries, 
	showtabs=false} 

\usepackage[norsk]{babel}
\usepackage{csquotes}
\usepackage{todonotes} 
\usepackage{subcaption}
\usepackage{comment}
\usepackage{wrapfig}

\title{Problem 1 - Use linear/quadratic functions for vertification}
\date{September 2019}
\author{Marte Fossum}

\begin{document}

{\setlength{\parindent}{0cm}

\maketitle
    
\section*{a}

We are given the equation and boundary conditions

\begin{align}\label{noe3}
	u'' + \omega^2u = f(t), \quad u(0) = 0, \quad u'(0) = V, \quad t \in [0, T]
\end{align}

To find \(u^1\) we use the numerical representation of the double derivative. 

\begin{align*}
	u'' \sim \frac{u^{n+1} - 2u^n + u^{n-1}}{(\Delta t)^2}
\end{align*}

Putting in this in the original equation gives us 

\begin{align*}
	\frac{u^{n+1} - 2u^n + u^{n-1}}{(\Delta t)^2} + \omega^2u = f^n 
\end{align*}

Solving this for \(u^{n+1}\) gives us 

\begin{align}\label{noe}
	u^{n+1} = f^n(\Delta t)^2 - \omega^2 u^n (\Delta t)^2 + 2u^n - u^{n-1} 
\end{align}

This equation however requires us to know \(u^{-1}\) which we do not. What we do then is use the initial condition \(u'(0) = V\) and approximate the derivative. We then get

\begin{align*}
	u' \sim \frac{u^{n+1} - u^{n-1}}{2\Delta t} 
\end{align*}

Setting in the condition \(t = 0\) we get that 

\begin{align*}
	u'(0) = \frac{u^1 - u^{-1}}{2\Delta t} = V
\end{align*}

We then again solve this for \(u^{-1}\) and get

\begin{align*}
	u^{-1} = u^1 - 2V\Delta t
\end{align*}

If we set \(n = 0\), our expression for \(u^{-1}\) and \(u^0\) into equation (\ref{noe}), we get an expression for \(u^1\) 

\begin{align*}
	u^1 &= f^0(\Delta t)^2 - \omega^2 u^0 (\Delta t)^2 + 2u^0 - u^{-1} \\
		&= \frac{1}{2}f^0(\Delta t)^2 - \frac{1}{2}\omega^2 I (\Delta t)^2 + I + V\Delta t
\end{align*}

\section*{b}

We are given the exact solution

\begin{align*}
	u_e(x,t) = ct + d
\end{align*}

From our bourdary condisions we get that \(u_e(x,0) = I\) which gives us that \(d = I\) and we have that \(u_{e}'(0) = V\) which gives us that \(c = V\). So now our exact equation looks like

\begin{align}\label{noe2}
	u_e(x,t) = Vt + I
\end{align}

From definition of the derivative we have that

\begin{align*}
	[D_t D_t t]^n &= \frac{[D_t t]^{n + \frac{1}{2}} - [D_t t]^{n - \frac{1}{2}}}{\Delta t} \\
				&= \cfrac{\cfrac{t^{n + \frac{1}{2} + \frac{1}{2}} - t^{n - \frac{1}{2} + \frac{1}{2}}}{\Delta t} - \cfrac{t^{t - \frac{1}{2} + \frac{1}{2}} - t^{n - \frac{1}{2} - \frac{1}{2}}}{\Delta t}}{\Delta t} \\
				&= \cfrac{t^{n+1} - t^n - t^n + t^{n-1}}{(\Delta t)^2} \\
				&= \cfrac{t^{n+1} - 2t^n + t^{n-1}}{(\Delta t)^2} \\
				&= \cfrac{t + \Delta t - 2t + t - \Delta t}{(\Delta t)^2} \\
				&= 0
\end{align*}

where I have used that \(t^{n+1} = (n + 1)\Delta t = t + \Delta t\).  \\

Now using that \(D_t D_t\) is a linear operator, that is,

\begin{align*}
	[D_t D_t (ct + d)]^n &= c[D_t D_t t]^n + [D_t D_t d]^n \\
						&= 0
\end{align*}

we want to show that our exact solution is a perfect solution of the discrete equations. \\

We can easily see that inserted \(t = 0\) into the exact solution (\ref{noe2}) and its derivative, we get the boudary conditions. Inserting the exact solution into (\ref{noe3}) gives us an expression for \(f(t)\)

\begin{align*}
	\omega^2 (Vt + I) = f(t)
\end{align*}

Now we insert the exact solution into the descrete equation. We have already shown that \(D_t D-t (Vt + I) = 0\) so we are left with 

\begin{align*}
	[\omega^2u_e(x,t) &= f(t)]^n \\
	[\omega^2(Vt + I) &= f(t)]^n
\end{align*}

which was what we found for (\ref{noe3}) as well. 
}
\end{document}
